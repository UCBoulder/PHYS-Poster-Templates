% CU Boulder Physics Department — Poster Template
% 35.61" x 47.48" landscape poster using beamerposter (fits 36" roll printers)
%
% Compile with:
%   pdflatex poster && bibtex poster && pdflatex poster && pdflatex poster

\documentclass[final]{beamer}
\usepackage[orientation=landscape, size=custom, width=120.60, height=90.45, scale=1.4]{beamerposter}
% Width=47.48in=120.60cm, Height=35.61in=90.45cm (fits 36" roll printers without scaling)
% scale=1.4 sizes fonts appropriately for poster printing

\usepackage[utf8]{inputenc}
\usepackage[T1]{fontenc}
\usepackage{helvet}
\renewcommand{\familydefault}{\sfdefault}
\usepackage{amsmath, amssymb}
\usepackage{graphicx}
\usepackage{booktabs}
\usepackage{multicol}
\usepackage{ragged2e}
\hypersetup{hidelinks}
\usepackage[backend=bibtex, style=numeric-comp, maxnames=3]{biblatex}
\addbibresource{references.bib}

% ============================================================
% === CONFIGURATION — EDIT THIS SECTION ===
% ============================================================

% Poster metadata
\newcommand{\postertitle}{Your Poster Title Goes Here}
\newcommand{\posterauthor}{Author One, Author Two, Author Three}
\newcommand{\posteraffiliation}{Department of Physics, University of Colorado Boulder}
\newcommand{\posteremail}{author@colorado.edu}

% Logo paths (relative to this file)
% To use only one logo, comment out the other with %
\newcommand{\logoleft}{../assets/Physics_rev_left.png}
\newcommand{\logoright}{../assets/Physics_rev_left.png}

% Footer content (comment out \footercontent to remove the footer entirely)
\newcommand{\footercontent}{
  \href{mailto:author@colorado.edu}{author@colorado.edu} \hfill
  ABC Conference 2025, City \hfill
  \href{https://www.colorado.edu}{https://www.colorado.edu}
}

% Column width: use 0.31\textwidth for 3 columns, 0.47\textwidth for 2 columns
\newlength{\colwidth}
\setlength{\colwidth}{0.31\textwidth}

% ============================================================
% === CU BOULDER COLORS ===
% ============================================================

\definecolor{CUGold}{HTML}{CFB87C}
\definecolor{CUBlack}{HTML}{000000}
\definecolor{CUDarkGray}{HTML}{565A5C}
\definecolor{CULightGray}{HTML}{A2A4A3}
\definecolor{CUSkyBlue}{HTML}{096FAE}
\definecolor{CUDarkBlue}{HTML}{0A3758}
\definecolor{CULightGold}{HTML}{F3F0E9}
\definecolor{CUAccessibleGold}{HTML}{8D7334}

% ============================================================
% === BEAMER THEME SETUP ===
% ============================================================

\usetheme{default}
\setbeamercolor{background canvas}{bg=white}

% Remove navigation symbols
\setbeamertemplate{navigation symbols}{}

% Title banner color
\setbeamercolor{banner}{bg=CUBlack, fg=white}
\setbeamercolor{gold bar}{bg=CUGold}

% Block styling — rounded corners, generous padding
\setbeamercolor{block title}{fg=white, bg=CUBlack}
\setbeamercolor{block body}{fg=CUBlack, bg=CULightGold}
\setbeamerfont{block title}{size=\LARGE, series=\bfseries}
\setbeamerfont{block body}{size=\large}
\setbeamertemplate{block begin}{
  \begin{beamercolorbox}[wd=\textwidth, rounded=true, shadow=false, leftskip=1cm, rightskip=1cm, colsep*=0.75cm]{block title}%
    \rule[-0.3cm]{0pt}{1.5cm}\usebeamerfont{block title}\insertblocktitle
  \end{beamercolorbox}%
  \nointerlineskip
  \begin{beamercolorbox}[wd=\textwidth, rounded=true, shadow=false, vmode, colsep*=0.5cm]{block body}%
    \vspace{0.3cm}
    \hspace{1cm}\begin{minipage}{\dimexpr\textwidth-2cm\relax}%
    \usebeamerfont{block body}%
}
\setbeamertemplate{block end}{
    \end{minipage}\hspace{1cm}%
    \vspace{0.5cm}
  \end{beamercolorbox}
}

% Alternative block (for highlights)
\setbeamercolor{block alerted title}{fg=white, bg=CUDarkBlue}
\setbeamercolor{block alerted body}{fg=CUBlack, bg=white}
\setbeamertemplate{block alerted begin}{
  \begin{beamercolorbox}[wd=\textwidth, rounded=true, shadow=false, leftskip=1cm, rightskip=1cm, colsep*=0.75cm]{block alerted title}%
    \rule[-0.3cm]{0pt}{1.5cm}\usebeamerfont{block title}\insertblocktitle
  \end{beamercolorbox}%
  \nointerlineskip
  \begin{beamercolorbox}[wd=\textwidth, rounded=true, shadow=false, vmode, colsep*=0.5cm]{block alerted body}%
    \vspace{0.3cm}
    \hspace{1cm}\begin{minipage}{\dimexpr\textwidth-2cm\relax}%
    \usebeamerfont{block body}%
}
\setbeamertemplate{block alerted end}{
    \end{minipage}\hspace{1cm}%
    \vspace{0.5cm}
  \end{beamercolorbox}
}

% Itemize styling
\setbeamercolor{item}{fg=CUGold}
\setbeamertemplate{itemize item}{\raise0.5pt\hbox{\vrule width 0.8ex height 0.8ex}}
\setbeamertemplate{itemize subitem}{\raise0.5pt\hbox{\vrule width 0.6ex height 0.6ex}}

% Footer banner (gold line + black bar, mirrors the title banner)
\setbeamertemplate{footline}{
  \ifdefined\footercontent
  \nointerlineskip
  \begin{beamercolorbox}[wd=\paperwidth, ht=8pt]{gold bar}
  \end{beamercolorbox}
  \nointerlineskip
  \begin{beamercolorbox}[wd=\paperwidth, leftskip=2cm, rightskip=2cm, colsep*=0.75cm]{banner}
    {\large\color{white}\footercontent\par}
  \end{beamercolorbox}
  \fi
}

% ============================================================
% === DOCUMENT ===
% ============================================================

\begin{document}
\begin{frame}[t]

% --- Title Banner (black background, white text, gold accent) ---
\begin{beamercolorbox}[wd=\paperwidth, leftskip=1.5cm, rightskip=1.5cm]{banner}
  \vspace{1.5cm}
  \begin{columns}[b]
    \begin{column}{0.20\textwidth}
      % Left logo — comment out \includegraphics to remove
      \hfill
      \ifdefined\logoleft\includegraphics[width=0.85\linewidth]{\logoleft}\fi
    \end{column}
    \begin{column}{0.60\textwidth}
      \centering
      {\VERYHuge\bfseries\color{white}\postertitle\par}
      \vspace{1cm}
      {\LARGE\color{CULightGray}\posterauthor\par}
      \vspace{0.5cm}
      {\Large\color{CULightGray}\posteraffiliation\par}
    \end{column}
    \begin{column}{0.20\textwidth}
      % Right logo — comment out \includegraphics to remove
      \ifdefined\logoright\includegraphics[width=0.85\linewidth]{\logoright}\fi
    \end{column}
  \end{columns}
  \vspace{1.5cm}
\end{beamercolorbox}
\nointerlineskip
\begin{beamercolorbox}[wd=\paperwidth, ht=8pt]{gold bar}
\end{beamercolorbox}

\vspace{1.5cm}

% --- Content Columns ---
% This template uses 3 columns. For a 2-column layout:
%   1. Change \colwidth above to 0.47\textwidth
%   2. Move the Column 3 content into Column 1 or Column 2
%   3. Delete the Column 3 section below
\begin{columns}[T]

% === COLUMN 1 ===
\begin{column}{\colwidth}

\begin{block}{Introduction}
  Provide the scientific context and motivation for your work. What problem are you
  addressing, and why does it matter?

  \vspace{0.5cm}

  State the broader context of the research area, explain why this particular question
  is important, and outline the approach you took to investigate it.

  \vspace{0.5cm}

  \begin{itemize}
    \item Background on the topic and current state of knowledge
    \item Key open question or hypothesis being tested
    \item Brief statement of your experimental or theoretical approach
    \item Overview of main findings (one sentence)
  \end{itemize}
\end{block}

\vspace{2cm}

\begin{block}{Methods / Experimental Setup}
  Describe your experimental apparatus, simulation framework, or analytical methods.

  \vspace{0.5cm}

  % Sample equation
  The relativistic energy-momentum relation is:
  \begin{equation}
    E^2 = (pc)^2 + (m_0 c^2)^2
    \label{eq:energy_momentum}
  \end{equation}

  \vspace{0.3cm}

  \begin{itemize}
    \item Equipment specifications and measurement techniques
    \item Calibration and systematic uncertainty estimation
    \item Data collection procedures and sample sizes
    \item Statistical or computational analysis methods
  \end{itemize}
\end{block}

\vfill

\end{column}

% === COLUMN 2 ===
\begin{column}{\colwidth}

\begin{block}{Results}
  Present your key findings with figures, plots, and tables.

  \vspace{0.5cm}

  % Sample figure placeholder — replace with your actual figure
  \begin{figure}
    \centering
    % \includegraphics[width=0.9\columnwidth]{your-figure.pdf}
    \fbox{\parbox{0.85\columnwidth}{\centering\vspace{8cm}
      {\LARGE\color{CUDarkGray}[Your figure here]}\\[1cm]
      {\large Replace this box with \texttt{\textbackslash includegraphics}}
    \vspace{8cm}}}
    \caption{Description of your figure. Use 300\,DPI images for best print quality.}
    \label{fig:result}
  \end{figure}

  \vspace{1cm}

  % Sample table
  \begin{table}
    \centering
    \caption{Sample measurement results.}
    \label{tab:results}
    \begin{tabular}{lcc}
      \toprule
      \textbf{Parameter} & \textbf{Value} & \textbf{Uncertainty} \\
      \midrule
      Mass (kg) & 1.6 & $\pm\,0.3$ \\
      Velocity (m/s) & 2.998 $\times 10^8$ & $\pm\,0.001 \times 10^8$ \\
      Energy (J) & 1.503 $\times 10^{17}$ & $\pm\,0.005 \times 10^{17}$ \\
      \bottomrule
    \end{tabular}
  \end{table}
\end{block}

\vfill

\end{column}

% === COLUMN 3 ===
% To switch to a 2-column layout, delete from here to "END COLUMN 3".
\begin{column}{\colwidth}

\begin{block}{Discussion / Conclusions}
  Interpret your results and discuss their significance.

  \vspace{0.5cm}

  \begin{itemize}
    \item Summary of key findings and their quantitative agreement
    \item Comparison with theoretical predictions or prior experimental results
    \item Discussion of systematic uncertainties and limitations
    \item Implications for the field and future directions
  \end{itemize}

  \vspace{0.5cm}

  As shown in Eq.~\eqref{eq:energy_momentum} and Fig.~\ref{fig:result},
  our measurements are consistent with the theoretical prediction to within
  the stated uncertainties.
\end{block}

\vspace{2cm}

\begin{block}{References}
  \printbibliography[heading=none]
\end{block}

\vspace{2cm}

\begin{block}{Acknowledgments}
  This work was supported by [funding agency/grant number].
  We thank [advisor, collaborators, lab group] for valuable discussions
  and support.
\end{block}

\vfill

\end{column}
% END COLUMN 3

\end{columns}

\end{frame}
\end{document}
